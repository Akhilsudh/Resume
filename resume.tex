\documentclass[10pt, letterpaper]{article}

% Packages:
\usepackage[
    ignoreheadfoot, % set margins without considering header and footer
    top=2 cm, % seperation between body and page edge from the top
    bottom=2 cm, % seperation between body and page edge from the bottom
    left=2 cm, % seperation between body and page edge from the left
    right=2 cm, % seperation between body and page edge from the right
    footskip=1.0 cm, % seperation between body and footer
    % showframe % for debugging 
]{geometry} % for adjusting page geometry
\usepackage{titlesec} % for customizing section titles
\usepackage{tabularx} % for making tables with fixed width columns
\usepackage{array} % tabularx requires this
\usepackage[dvipsnames]{xcolor} % for coloring text
\definecolor{primaryColor}{RGB}{0, 0, 0} % define primary color
\usepackage{enumitem} % for customizing lists
\usepackage[dvipsnames]{xcolor}
\usepackage{fontawesome5} % for using icons
\usepackage{amsmath} % for math
\usepackage[
    pdftitle={Akhil Sudhakaran's CV},
    pdfauthor={Akhil Sudhakaran},
    pdfcreator={LaTeX with RenderCV},
    colorlinks=true,
    urlcolor=primaryColor
]{hyperref} % for links, metadata and bookmarks
\usepackage[pscoord]{eso-pic} % for floating text on the page
\usepackage{calc} % for calculating lengths
\usepackage{bookmark} % for bookmarks
\usepackage{lastpage} % for getting the total number of pages
\usepackage{changepage} % for one column entries (adjustwidth environment)
\usepackage{paracol} % for two and three column entries
\usepackage{ifthen} % for conditional statements
\usepackage{needspace} % for avoiding page brake right after the section title
\usepackage{iftex} % check if engine is pdflatex, xetex or luatex

% Ensure that generate pdf is machine readable/ATS parsable:
\ifPDFTeX
    \input{glyphtounicode}
    \pdfgentounicode=1
    \usepackage[T1]{fontenc}
    \usepackage[utf8]{inputenc}
    \usepackage{lmodern}
\fi

\usepackage{charter}

% Some settings:
\raggedright
\AtBeginEnvironment{adjustwidth}{\partopsep0pt} % remove space before adjustwidth environment
\pagestyle{empty} % no header or footer
\setcounter{secnumdepth}{0} % no section numbering
\setlength{\parindent}{0pt} % no indentation
\setlength{\topskip}{0pt} % no top skip
\setlength{\columnsep}{0.15cm} % set column seperation
\pagenumbering{gobble} % no page numbering

\titleformat{\section}{\needspace{4\baselineskip}\bfseries\large}{}{0pt}{}[\vspace{1pt}\titlerule]

\titlespacing{\section}{
    % left space:
    -1pt
}{
    % top space:
    0.3 cm
}{
    % bottom space:
    0.2 cm
} % section title spacing

\renewcommand\labelitemi{$\vcenter{\hbox{\small$\bullet$}}$} % custom bullet points
\newenvironment{highlights}{
    \begin{itemize}[
        topsep=0.10 cm,
        parsep=0.10 cm,
        partopsep=0pt,
        itemsep=0pt,
        leftmargin=0 cm + 10pt
    ]
}{
    \end{itemize}
} % new environment for highlights


\newenvironment{highlightsforbulletentries}{
    \begin{itemize}[
        topsep=0.10 cm,
        parsep=0.10 cm,
        partopsep=0pt,
        itemsep=0pt,
        leftmargin=10pt
    ]
}{
    \end{itemize}
} % new environment for highlights for bullet entries

\newenvironment{onecolentry}{
    \begin{adjustwidth}{
        0 cm + 0.00001 cm
    }{
        0 cm + 0.00001 cm
    }
}{
    \end{adjustwidth}
} % new environment for one column entries

\newenvironment{twocolentry}[2][]{
    \onecolentry
    \def\secondColumn{#2}
    \setcolumnwidth{\fill, 4.5 cm}
    \begin{paracol}{2}
}{
    \switchcolumn \raggedleft \secondColumn
    \end{paracol}
    \endonecolentry
} % new environment for two column entries

\newenvironment{threecolentry}[3][]{
    \onecolentry
    \def\thirdColumn{#3}
    \setcolumnwidth{, \fill, 4.5 cm}
    \begin{paracol}{3}
    {\raggedright #2} \switchcolumn
}{
    \switchcolumn \raggedleft \thirdColumn
    \end{paracol}
    \endonecolentry
} % new environment for three column entries

\newenvironment{header}{
    \setlength{\topsep}{0pt}\par\kern\topsep\centering\linespread{1.5}
}{
    \par\kern\topsep
} % new environment for the header

\newcommand{\placelastupdatedtext}{% \placetextbox{<horizontal pos>}{<vertical pos>}{<stuff>}
  \AddToShipoutPictureFG*{% Add <stuff> to current page foreground
    \put(
        \LenToUnit{\paperwidth-2 cm-0 cm+0.05cm},
        \LenToUnit{\paperheight-1.0 cm}
    ){\vtop{{\null}\makebox[0pt][c]{
        \small\color{gray}\textit{Last updated in September 2024}\hspace{\widthof{Last updated in September 2024}}
    }}}%
  }%
}%

% save the original href command in a new command:
\let\hrefWithoutArrow\href

% new command for external links:


\begin{document}
    \newcommand{\AND}{\unskip
        \cleaders\copy\ANDbox\hskip\wd\ANDbox
        \ignorespaces
    }
    \newsavebox\ANDbox
    \sbox\ANDbox{$|$}

    \begin{header}
        \fontsize{25 pt}{25 pt}\selectfont 
        \textbf{{\color{MidnightBlue} Akhil} Sudhakaran}

        \vspace{3 pt}

        \normalsize
        \mbox{Bengaluru}%
        \kern 3.0 pt%
        \AND%
        \kern 3.0 pt%
        \mbox{ \hrefWithoutArrow{mailto:akhil.sudh@gmail.com}{akhil.sudh@gmail.com}}%
        \kern 3.0 pt%
        \AND%
        \kern 3.0 pt%
        \mbox{ \hrefWithoutArrow{tel:+919886395402}{9886395402}}%
        \kern 3.0 pt%
        \AND%
        \kern 3.0 pt%
        \mbox{\hrefWithoutArrow{https://akhilsudh.github.io/}{akhilsudh.github.io}}%
        \kern 3.0 pt%
        \AND%
        \kern 3.0 pt%
        \mbox{\hrefWithoutArrow{https://linkedin.com/in/akhilsudh/}{linkedin/akhilsudh}}%
        \kern 3.0 pt%
        % \AND%
        % \kern 3.0 pt%
        % \mbox{\hrefWithoutArrow{https://github.com/akhilsudh}{github/akhilsudh}}%
    \end{header}

    \vspace{5 pt}

    \section{Professional Summary}    
        I am a senior software developer currently working at OpenText, Bengaluru. With over 6 years of working experience as an individual contributor, I offer technical expertise in designing, developing and maintaining scalable Springboot Java applications, microservices and interactive React web user interfaces for the IT Operations Management (ITOM) suite. I have a proven track record of delivering software in a SAFe® Agile release cycle with high velocity. Additionally I have experience in handling live customer defect debugging and resolution.

    \section{Work History}
        
        \begin{twocolentry}{
            \textbf{Apr 2024 – Present}
        }
            {\textbf{{\color{MidnightBlue}Software Development Engineer III} | OpenText -- Bengaluru, India}}\end{twocolentry}

        \vspace{0.20 cm}
        \begin{onecolentry}
            \textbf{Service Management Automation X (SMAX)}
            \newline The SMAX platform manages IT Services, Assets and Enterprise services using system apps. It uses Java for backend and AngularJS for the WebUI. Currently \textbf{I am working in the platform team where we take ownership, design and implement backend features}.
            \begin{highlights}
                \item Worked on the image metadata redaction logic for the image upload service. Replaced a dependency on a 3rd party image library for custom image save \textbf{leading to reduced disk space consumption for tif images by 30\%}.
                \item Added \textbf{Redis caching for API services to improve response time by reducing DB calls by 40\%}.
                \item Designed and \textbf{created a Spring microservice}, that correlates Vulnerability and CVE data from Qualys data source with the vulnerable devices scanned by Server Automation (An OpenText ITOM solution).
                \item \textbf{Implemented transaction management across microservices} involved in Vulnerability scan and remediation.
                \item \textbf{Implemented Row Level Access Control (RLAC)} on the backend to enable user access control over specific rows of all entities in the DB based on predetermined rules set by the admin.
                \item \textbf{Increased JUnit/JBehave UT coverage from 40\% to over 80\%} across modules owned by Bangalore team.
                \item Contributed in \textbf{over 50 code reviews per PI} and in \textbf{customer documentation for the product}.
            \end{highlights}
        \end{onecolentry}
        
        \vspace{0.20 cm}
        \begin{onecolentry}
            \textbf{Operations Orchestrator (OO)}
            \newline OO is a workflow orchestrator that is based on Java backend for the orchestration engine and React for the UI and workflow designer web tool. \textbf{I Worked with the OO team to deliver UI features on OO Designer} (A Cloudslang \& AFL workflow creation tool).
            \begin{highlights}
                \item Implemented \textbf{drag and drop functionality} for workflow items directly on the web UI using \textbf{React, Redux \& mxGraph}, effectively \textbf{improving workflow creation speed by 100\%}.
                \item Developed \textbf{real-time workflow validation} on importing/authoring workflows on the web UI or by Git changes.
                \item Created \textbf{UI test automation using Selenium, reducing manual test efforts by 5 story points per sprint and increasing QA efficiency by 50\%} and contributed in defect fixing and product documentation.
            \end{highlights}
        \end{onecolentry}


        \vspace{0.5 cm}

        \begin{twocolentry}{
            \textbf{Sep 2021 – Mar 2024}
        }
            {\textbf{{\color{MidnightBlue}Software Development Engineer II} |  OpenText -- Bengaluru, India}}\end{twocolentry}
        \vspace{0.20 cm}
        \begin{onecolentry}
            \textbf{Database and Middleware Automation (DMA)}
            \newline DMA automates deploying, managing and patching DB and Middleware applications thorough workflows. I worked in the DMA platform team, where I was responsible for \textbf{modernizing the legacy code base}.
            \begin{highlights}
                \item \textbf{Upgraded the platform code from Java 8 to Java 11}, improved the encryption logic from AES to PKCS-GCM for workflow data and fixed other security defects.
                \item \textbf{Removed dependency on deprecated Jython interpreter by upgrading to python 3} for the workflow execution engine and \textbf{rewrote the workflows in python 3}.
                \item Replaced \textbf{Adobe Flash for UI elements to use charts.js} (A modern Javascript based charting library) \textbf{increasing dashboard responsiveness by a 100\%}.
                \item \textbf{Fixed over 20 defects from the backlog}, engaged in several customer facing defect resolution calls and contributed in comprehensive upgrade documentation.
            \end{highlights}
        \end{onecolentry}

         \vspace{0.20 cm}
        \begin{onecolentry}
            \textbf{Data Center Automation (DCA)}
            \newline DCA is a standalone application that automates vulnerability and compliance scan for private cloud data centers.
            \begin{highlights}
                \item \textbf{Optimized memory resource allocation} for report queries \textbf{from 2GB to 100MB} on Vertica DB
                \item Worked on the backend code for \textbf{internationalization and localization platform features}.
                \item Wrote \textbf{scripts to import patching data from WSUS and RedHat Satellite} for DCA patch scan capability.
            \end{highlights}
        \end{onecolentry}

        \vspace{0.5 cm}

        \begin{twocolentry}{
            \textbf{Aug 2018 – Aug 2021}
        }
            {\textbf{{\color{MidnightBlue}Software Development Engineer I} |  Micro Focus -- Bengaluru, India}}\end{twocolentry}
        \vspace{0.10 cm}

        \vspace{0.10 cm}
         \begin{onecolentry}
            \begin{highlights}
                \item Worked on features for a \textbf{multithreaded java application} called dataminer for extracting, transforming and loading data (ETL) from the Server automation platform DB to the in house reporting solution (OPTIC DL).
                \item Took ownership of OO content pack for Server Automation and worked on it's features and security defects.
                \item Wrote Ansible playbooks for deploying dataminer on SA as an \textbf{RnD efficiency item}.
                \item Responsible for \textbf{converting Python2 code in vulnerability content to Python3} as part of modernizing the code.
                \item Worked with the content team to \textbf{deliver vulnerability and compliance content in OVAL format}.
                \item Created \textbf{automation in python using the in-house AXIS framework} for content testing and deploying.
            \end{highlights}
        \end{onecolentry}

    \section{Technologies}    
        \begin{onecolentry}
            \textbf{Languages:} Java, Python, SQL, Javascript/Typescript, HTML/CSS, shell scripting
        \end{onecolentry}
        \begin{onecolentry}
            \textbf{Frameworks:} SpringBoot, Hibernate(JPA), Junit(TDD), JBehave(BDD), React, Selenium
        \end{onecolentry}
        \begin{onecolentry}
            \textbf{Miscellaneous:} Git, Maven, Postgres, Vertica, Redis, Kafka, SAFe® Agile, Microservices, Linux, Windows, WSL, Jenkins, Docker, Kubernetes, AWS
        \end{onecolentry}

    \section{Education}        
        \begin{twocolentry}{
            \textbf{2021 – 2023}
        }
            \textbf{{\color{MidnightBlue}Birla Institute of Technology, Pilani}, MTech in System Software}\end{twocolentry}

        \vspace{0.10 cm}
        \begin{onecolentry}
            \begin{highlights}
                \item \textbf{CGPA:} 9.24
                \item \textbf{Related Coursework:} Distributed Computing, Mathematical Foundations for Data Science, Advanced Statistics, Applied Machine Learning, Deep Learning, Natural Language Processing.
            \end{highlights}
        \end{onecolentry}

        \vspace{0.50 cm}

        \begin{twocolentry}{
            \textbf{2014 – 2018}
        }
            \textbf{{\color{MidnightBlue}Sir M. Visvesvaraya Institute of Technology, Bengaluru}, BE in Computer Science}\end{twocolentry}

        \vspace{0.10 cm}
        \begin{onecolentry}
            \begin{highlights}
                \item \textbf{Grade:} 80.5%
                \item \textbf{Related Coursework:} Object Oriented programming with Java, Pattern Recognition, Artificial Intelligence, Network Security, Software Testing, Unix system programming.  
            \end{highlights}
        \end{onecolentry}

    \section{Achievements and Publications}
        \begin{highlights}        
        \item Awarded one of the \textbf{best RnD teams in Bangalore} for the year 2023-2024.
        \item One among the \textbf{top 10\% of my class in BITS Pilani} for the MTech batch of 2021-23.
        \item Recognized for innovations done as part of my contributions in inno weeks.
        \item \textbf{SAFe® Certified} Agile Developer 2018
        \item \textbf{Awarded gold medal for being the top 5\%} of FCD holders of Sir. M Visvesvaraya Institute of Technology, Bengaluru for the BE batch of 2014-18
        \item Received 1 among the 1000 scholarships given nationwide by Google and Tata Trust for the Udacity Android Development Nanodegree. 
        \item \begin{samepage}
            \begin{twocolentry}{
                \textbf{Jul 2017}
            }
                \textbf{A Novel Double Backtracking Approach to the
 N-Queens Problem in Three Dimensions}
            \end{twocolentry}

            \vspace{0.10 cm}
            
            \begin{onecolentry}
                \mbox{Abhijith Chakiat}, \mbox{\textbf{\textit{Akhil Sudhakaran}}}, \mbox{Abhinav A. Nair}, \mbox{Pallavi Venkatesh}

                \vspace{0.10 cm}
                
        \href{https://doi.org/10.5120/ijca2017914749}{10.5120/ijca2017914749}
        \end{onecolentry}
        \end{samepage}
        \end{highlights}
\end{document}